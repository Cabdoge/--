\documentclass[lang=cn,a4paper,11pt,openany,scheme=chinese,twoside]{elegantbook}

\let\arrowvert\relax
%\usepackage{unicode-math}

\let\arrowvert\relax

\usepackage{ctex}
\usepackage{xeCJK}
\usepackage{amsmath} % 建议
\usepackage{amssymb} % 可选,提供更多符号

\usepackage{ctex}
\let\openbox\relax

\usepackage{tikz}
\usetikzlibrary{decorations.pathreplacing,arrows.meta}

%调整页边距
\usepackage{geometry}
\geometry{top=2cm, bottom=2.25cm, outer=2cm, inner=2.5cm}


\usepackage{setspace}
\setstretch{1.3}

%设定行间距
\usepackage{calc}
\setlength{\lineskip}{\baselineskip-\ccwd}
\setlength{\lineskiplimit}{2.5pt}



%浮动图片
\usepackage{wrapfig}
%图标表格注记
\usepackage{caption}
%圆圈数字
\usepackage{pifont}
%表格换行
\usepackage{makecell}
%表格颜色
\usepackage[table]{xcolor}
\colorlet{rowcolcolor}{structurecolor!25!white}

%行间距加大
\newcommand{\wideline}{\setlength{\lineskip}{\baselineskip-\ccwd}\setlength{\lineskiplimit}{2.5pt}}
%蓝色加粗
\definecolor{titleblue}{HTML}{3498DB}
\newcommand{\bluebf}[1]{\textcolor{structurecolor}{\textbf{#1}}}


%向量
\usepackage[f]{esvect}

\renewcommand{\vector}[1]{\vv{#1}}
%平面
\newcommand{\plane}{\text{平面}\ }
%平行且等于
\newcommand{\paraleq}{%
    \mathrel{\text{%
        \tikz[baseline]
        \draw (.1em,0ex) -- (.9em,0ex)
        (.1em,-.425ex) -- (.9em,-.425ex)
        (.350em,.1ex) -- (.650em,1.5ex)
        (.550em,.1ex) -- (.850em,1.5ex);}}}

\newcommand{\mb}{\mathbb}
\newcommand{\z}{\text}
\renewcommand{\a}{\alpha}
\renewcommand{\b}{\beta}
\renewcommand{\d}{\delta}
\newcommand{\D}{\Delta}
\renewcommand{\l}{\lambda}
\newcommand{\m}{\mu}
\newcommand{\fl}{\fillin}
\newcommand{\pr}{\paren}
\newcommand{\p}{\uppi}
\renewcommand{\th}{\theta}
\newcommand{\lam}{\lambda}
\newcommand{\rttri}{\mathrm{Rt}\triangle}
\newcommand{\npar}{\par\noindent}
\DeclareSymbolFont{ugmL}{OMX}{mdugm}{m}{n}
\DeclareMathAccent{\wideparen}{\mathord}{ugmL}{"F3}
\newcommand{\sigmaalg}{\sigma-\text{代数}}
\newcommand{\setjot}[1][0]{\setlength{\jot}{#1 pt}}

\DeclareMathOperator{\st}{\text{s.t.}}
\DeclareMathOperator{\aew}{\text{a.e.}}


%填空题下划线
\newcommand{\fillin}[1][3.5]{%
  \nolinebreak % 不在此处断行
  \hspace{0.2em}%
  \rule[-0.5ex]{#1em}{0.5pt}% 轻微下沉模拟“下划线”效果
  \hspace{0.05em}
}

%调整环境内图片环绕
\usepackage{mwe}
\newcommand{\wrapfix}[1][-1]{\begin{wrapfigure}{r}{0.001\textwidth}%
    \vspace{#1 em}%
    \includegraphics[width=0.001\textwidth]{example-image}%
    \end{wrapfigure}}


%分式使用行间版
\everymath{\displaystyle}


\usepackage{myenvironment}


%elegantbook部分修改
\usepackage{elegantfix}

%强制右页开始
\makeatletter
\newcommand{\ForceRightPage}{%
  \clearpage
  \ifodd\value{page}\relax
    % 已经是奇数页,什么也不做
  \else
    \hbox{}\thispagestyle{empty}\newpage
  \fi
}
\makeatother

\raggedbottom


\begin{document}


\title{基础拓扑学笔记}
\author{庞逸天}
\date{}
\maketitle
\setcounter{page}{1}

\frontmatter
\pagenumbering{roman}     % 目录等用罗马数字
\clearpage
\tableofcontents
\clearpage

\ForceRightPage
\mainmatter

\chapter{集合论复习}

\begin{definition}{幂集}
    设$X$是一个集合,$X$的所有子集组成的集合称为$X$的幂集,记作$2^X$.
\end{definition}

\begin{definition}{集合的卡氏积}
    设$X_1,X_2,\cdots,X_n$为集合,则它们的卡氏积定义为$X_1\times X_2\times\cdots\times X_n := \big\{(x_1,x_2,\cdots,x_n)|x_i\in X_i\big\}$.
\end{definition}

\begin{proposition}{集合运算的分配律}
    \wideline
    \begin{enumerate}
        \item $A\cup \left(\bigcap_{\lambda\in\Lambda}B_\lambda\right)=\bigcap_{\lambda\in\Lambda}{\left(A\cup B_\lambda\right)}$
        \item $A\cap \left(\bigcup_{\lambda\in\Lambda}B_\lambda\right)=\bigcup_{\lambda\in\Lambda}{\left(A\cap B_\lambda\right)}$
        \item $A\times \left(\bigcup_{\lambda\in\Lambda}B_\lambda\right)=\bigcup_{\lambda\in\Lambda}{\left(A\times B_\lambda\right)}$
        \item $A\times \left(\bigcap_{\lambda\in\Lambda}B_\lambda\right)=\bigcap_{\lambda\in\Lambda}{\left(A\times B_\lambda\right)}$
        \item $A\times \left(B_1\setminus B_2\right)=(A\times B_1)\setminus (A\times B_2)$
    \end{enumerate}
\end{proposition}

\begin{theorem}{$\text{德}\cdot\text{摩根定律}$}
    \wideline
    \begin{enumerate}
        \item $A\setminus \left(\bigcup_{\lambda\in\Lambda}{B_{\lambda}}\right)=\bigcap_{\lambda\in\Lambda}{\left(A\setminus B_{\lambda}\right)}$
        \item $A\setminus \left(\bigcap_{\lambda\in\Lambda}{B_{\lambda}}\right)=\bigcup_{\lambda\in\Lambda}{\left(A\setminus B_{\lambda}\right)}$
    \end{enumerate}
\end{theorem}

\begin{definition}{映射}
    设$X,Y$是两个集合,则$X,Y$之间的\textbf{映射}$f:X\to Y$将$X$中的每个元素$x$唯一地对应到$Y$中一个元素$y$.其中$X$称为\textbf{定义域},$Y$称为\textbf{值域}.
\end{definition}

\begin{definition}{映射的像}
    设$f:X\to Y$是一个映射,则子集$A\subseteq X$在$f$之下的像为
    \[f(A)=\big\{y\in Y\big|\exists\, x\in A,\,\st f(x)=y \big\}\subseteq Y\]
\end{definition}

\begin{definition}{原像}
    设$f:X\to Y$是一个映射,则子集$B\subseteq Y$在$f$之下的原像为
    \[f^{-1}(B)=\big\{x\in X\big|f(x)\in B \big\}\subseteq X\]
\end{definition}

\begin{note}
    注意,这里的$f^{-1}$不表示$f$的逆映射或是$f$的反函数.
\end{note}

\begin{definition}{逆映射}
    如果$f:X\to Y$是一个双射,定义$f^{-1}:Y\to X$是$f$的逆映射,满足$f:x\mapsto y \Rightarrow f^{-1}:y\mapsto x$.
\end{definition}

\begin{proposition}{映射的性质}
    \wideline
    如果$f:X\to Y$,$g:Y\to Z$是两个映射,$A_\lambda\subset X,B_\lambda \subset Y$,则
    \begin{enumerate}
        \item $f^{-1}\left(\bigcup_{\lambda}{B_\lambda}\right)=\bigcup_{\lambda}{f^{-1}(B_\lambda)}$
        \item $f^{-1}\left(\bigcap_{\lambda}{B_\lambda}\right)=\bigcap_{\lambda}{f^{-1}(B_\lambda)}$
        \item $f\left(\bigcup_{\lambda}{B_\lambda}\right)=\bigcup_{\lambda}{f(B_\lambda)}$
        \item $f\left(\bigcap_{\lambda}{B_\lambda}\right)\subset\bigcap_{\lambda}{f(B_\lambda)}$,当$f$为单射时,等号成立.
        \item $f\left(f^{-1}(B)\right)\subset B$,当$f$为满射时,等号成立.
        \item $f^{-1}\left(f(A)\right)\supset A$,当$f$为单射时,等号成立.
        \item $(g\circ f)(A)=g \left(f(A)\right),A\subset X$
        \item $(g\circ f)^{-1}(C)=f^{-1}\left(g^{-1}(C)\right),C\subset Z$
    \end{enumerate}
\end{proposition}


\begin{definition}{映射的限制}
    设$f:X\to Y$是一个映射,$A\subset X$,定义$f$在$A$上的限制为
    \begin{align*}
        f|_A :A &\to F \\
        a &\mapsto f(a)
    \end{align*}
    反过来,$f$称为$f|_A$在$X$上的扩张.
\end{definition}

\begin{definition}{特殊的映射}
    \begin{enumerate}
        \item ${\setjot[-2]\begin{aligned}
            \mathrm{id}_X: X &\to X \\
            x &\mapsto x
        \end{aligned}}$称为$X$上的恒等映射.
        \item 若$A\subset X$,则${\setjot[-2]\begin{aligned}
            i:A &\to X \\
            a &\mapsto a
        \end{aligned}}$称为$A$到$X$的包含映射.
        \item $\setjot[-2]\begin{aligned}
            \Delta_X:X &\to X\times X \\
            x &\mapsto (x,x)
        \end{aligned}$称为$X$的对角映射.
    \end{enumerate}
\end{definition}

\begin{remark}
    关于包含映射,我们有以下交换图: \par
    \begin{tikzcd}[column sep=3.2em, row sep=2.4em,every label/.append style={font=\large},font=\large]
A \arrow[rr, "f|_A"] \arrow[rd, "i"'] &                    & Y \\
                                      & X \arrow[ru, "f"'] &  
\end{tikzcd}\par

也就是说,我们可以把限制映射看成包含映射和映射的复合.

\begin{definition}{等价关系}
    设$\sim$是集合$X$上的一个二元关系,如果它满足以下条件:
    \begin{enumerate}
        \item 自反性:$\forall\, x\in X,\,x\sim x$
        \item 对称性: $\forall\, x,y\in X,\,x\sim y\Rightarrow y\sim x$
        \item 传递性: $\forall\, x,y,z\in X,\,x\sim y\,\text{且}\,y\sim z\Rightarrow x\sim z$
    \end{enumerate}
    则称$\sim$为$X$上的一个等价关系.
\end{definition}

\begin{definition}{等价类}
    设$\sim$是集合$X$上的一个等价关系,则$A$中所有等价于$x$的元素组成的集合称为$x$的等价类,记作$[x]=\{y\in X|x\sim y\}$.
\end{definition}

\begin{proposition}{等价类的性质}
    两个等价类要么是相等的,要么是不交的.
\end{proposition}

\begin{definition}{商集}
    设$\sim$是集合$X$上的一个等价关系,所有等价类的集合称为商集,记为$X/\sim \;= \left\{[x]|x\in X\right\}$.
\end{definition}





\end{remark}

\end{document}
\documentclass[lang=cn,a4paper,11pt,openany,scheme=chinese,twoside]{elegantbook}

\let\arrowvert\relax
%\usepackage{unicode-math}

\let\arrowvert\relax

\usepackage{ctex}
\usepackage{xeCJK}

 %\setCJKfamilyfont{fangsong}{FangSong}[
  %BoldFont= FZHTK      % ← 关键!指定粗体字体
%]
%\renewcommand{\fangsong}{\CJKfamily{fangsong}}
\let\fangsong\relax
\newCJKfontfamily\fangsong{FangSong}[AutoFakeBold=2]

\let\kaishu\relax
\newCJKfontfamily\kaishu{KaiTi}[AutoFakeBold=2]
 

\usepackage{amsmath} % 建议
\usepackage{amssymb} % 可选,提供更多符号

\usepackage{ctex}
\let\openbox\relax

\usepackage{tikz}
\usetikzlibrary{decorations.pathreplacing,arrows.meta}

\usepackage{graphicx}

%商集符号
\usepackage{faktor}

%调整页边距
\usepackage{geometry}
\geometry{top=2cm, bottom=2.25cm, outer=2cm, inner=2.5cm}


\usepackage{setspace}
\setstretch{1.3}

%设定行间距
\usepackage{calc}
\setlength{\lineskip}{\baselineskip-\ccwd}
\setlength{\lineskiplimit}{2.5pt}



%浮动图片
\usepackage{wrapfig}
%图标表格注记
\usepackage{caption}
%圆圈数字
\usepackage{pifont}
%表格换行
\usepackage{makecell}
%表格颜色
\usepackage[table]{xcolor}
\colorlet{rowcolcolor}{structurecolor!25!white}

%行间距加大
\newcommand{\wideline}{\setlength{\lineskip}{\baselineskip-\ccwd}\setlength{\lineskiplimit}{2.5pt}}
%蓝色加粗
\definecolor{titleblue}{HTML}{3498DB}
\newcommand{\bluebf}[1]{\textcolor{structurecolor}{\textbf{#1}}}


%向量
\usepackage[f]{esvect}

%交换图
\usetikzlibrary{cd}

\renewcommand{\vector}[1]{\vv{#1}}
%平面
\newcommand{\plane}{\text{平面}\ }
%平行且等于
\newcommand{\paraleq}{%
    \mathrel{\text{%
        \tikz[baseline]
        \draw (.1em,0ex) -- (.9em,0ex)
        (.1em,-.425ex) -- (.9em,-.425ex)
        (.350em,.1ex) -- (.650em,1.5ex)
        (.550em,.1ex) -- (.850em,1.5ex);}}}

\newcommand{\mb}{\mathbb}
\newcommand{\z}{\text}
\renewcommand{\a}{\alpha}
\renewcommand{\b}{\beta}
\renewcommand{\d}{\delta}
\newcommand{\D}{\Delta}
\renewcommand{\l}{\lambda}
\newcommand{\m}{\mu}
\newcommand{\fl}{\fillin}
\newcommand{\pr}{\paren}
\newcommand{\p}{\uppi}
\renewcommand{\th}{\theta}
\newcommand{\lam}{\lambda}
\newcommand{\rttri}{\mathrm{Rt}\triangle}
\newcommand{\npar}{\par\noindent}
\DeclareSymbolFont{ugmL}{OMX}{mdugm}{m}{n}
\DeclareMathAccent{\wideparen}{\mathord}{ugmL}{"F3}
\newcommand{\sigmaalg}{\sigma-\text{代数}}
\newcommand{\setjot}[1][0]{\setlength{\jot}{#1 pt}}
\renewcommand{\cal}[1]{\mathcal{#1}}

\DeclareMathOperator{\st}{\text{s.t.}}
\DeclareMathOperator{\aew}{\text{a.e.}}


%填空题下划线
\newcommand{\fillin}[1][3.5]{%
  \nolinebreak % 不在此处断行
  \hspace{0.2em}%
  \rule[-0.5ex]{#1em}{0.5pt}% 轻微下沉模拟“下划线”效果
  \hspace{0.05em}
}

%调整环境内图片环绕
\usepackage{mwe}
\newcommand{\wrapfix}[1][-1]{\begin{wrapfigure}{r}{0.001\textwidth}%
    \vspace{#1 em}%
    \includegraphics[width=0.001\textwidth]{example-image}%
    \end{wrapfigure}}


%分式使用行间版
\everymath{\displaystyle}


\usepackage{myenvironment}


%elegantbook部分修改
\usepackage{elegantfix}

%强制右页开始
\makeatletter
\newcommand{\ForceRightPage}{%
  \clearpage
  \ifodd\value{page}\relax
    % 已经是奇数页,什么也不做
  \else
    \hbox{}\thispagestyle{empty}\newpage
  \fi
}
\makeatother

\raggedbottom


\begin{document}


\title{基础拓扑学笔记}
\date{}
\maketitle
\setcounter{page}{1}

\frontmatter
\pagenumbering{roman}     % 目录等用罗马数字
\clearpage
\tableofcontents
\clearpage

\ForceRightPage
\mainmatter

\chapter{集合论复习}

\begin{definition}{幂集}
    设$X$是一个集合,$X$的所有子集组成的集合称为$X$的幂集,记作$2^X$.
\end{definition}

\begin{definition}{集合的卡氏积}
    设$X_1,X_2,\cdots,X_n$为集合,则它们的卡氏积定义为$X_1\times X_2\times\cdots\times X_n := \big\{(x_1,x_2,\cdots,x_n)|x_i\in X_i\big\}$.
\end{definition}

\begin{proposition}{集合运算的分配律}
    \wideline
    \begin{enumerate}
        \item $A\cup \left(\bigcap_{\lambda\in\Lambda}B_\lambda\right)=\bigcap_{\lambda\in\Lambda}{\left(A\cup B_\lambda\right)}$
        \item $A\cap \left(\bigcup_{\lambda\in\Lambda}B_\lambda\right)=\bigcup_{\lambda\in\Lambda}{\left(A\cap B_\lambda\right)}$
        \item $A\times \left(\bigcup_{\lambda\in\Lambda}B_\lambda\right)=\bigcup_{\lambda\in\Lambda}{\left(A\times B_\lambda\right)}$
        \item $A\times \left(\bigcap_{\lambda\in\Lambda}B_\lambda\right)=\bigcap_{\lambda\in\Lambda}{\left(A\times B_\lambda\right)}$
        \item $A\times \left(B_1\setminus B_2\right)=(A\times B_1)\setminus (A\times B_2)$
    \end{enumerate}
\end{proposition}

\begin{theorem}{$\text{德}\cdot\text{摩根定律}$}
    \wideline
    \begin{enumerate}
        \item $A\setminus \left(\bigcup_{\lambda\in\Lambda}{B_{\lambda}}\right)=\bigcap_{\lambda\in\Lambda}{\left(A\setminus B_{\lambda}\right)}$
        \item $A\setminus \left(\bigcap_{\lambda\in\Lambda}{B_{\lambda}}\right)=\bigcup_{\lambda\in\Lambda}{\left(A\setminus B_{\lambda}\right)}$
    \end{enumerate}
\end{theorem}

\begin{definition}{映射}
    设$X,Y$是两个集合,则$X,Y$之间的\textbf{映射}$f:X\to Y$将$X$中的每个元素$x$唯一地对应到$Y$中一个元素$y$.其中$X$称为\textbf{定义域},$Y$称为\textbf{值域}.
\end{definition}

\begin{definition}{映射的像}
    设$f:X\to Y$是一个映射,则子集$A\subseteq X$在$f$之下的像为
    \[f(A)=\big\{y\in Y\big|\exists\, x\in A,\,\st f(x)=y \big\}\subseteq Y\]
\end{definition}

\begin{definition}{原像}
    设$f:X\to Y$是一个映射,则子集$B\subseteq Y$在$f$之下的原像为
    \[f^{-1}(B)=\big\{x\in X\big|f(x)\in B \big\}\subseteq X\]
\end{definition}

\begin{note}
    注意,这里的$f^{-1}$不表示$f$的逆映射或是$f$的反函数.
\end{note}

\begin{definition}{逆映射}
    如果$f:X\to Y$是一个双射,定义$f^{-1}:Y\to X$是$f$的逆映射,满足$f:x\mapsto y \Rightarrow f^{-1}:y\mapsto x$.
\end{definition}

\begin{definition}{}
    
\end{definition}

\chapter{拓扑空间}

\section{拓扑空间}

复习一下连续函数的相关知识.

\begin{theorem}{}
    $f:\mathbb{R}\to\mathbb{R}$是连续函数当且仅当任何开集的原像是开集.
\end{theorem}


\begin{proposition}{实数集上开集的性质}
    \begin{enumerate}
        \item $\varnothing$和$\mathbb{R}$是开集.
        \item 任意多个开集的并还是开集.
        \item 任意有限多个开集的交还是开集.
    \end{enumerate}
\end{proposition}
\begin{proof}
    第三条性质的证明:设$U_1=\bigcup_\alpha{(a_\alpha,b_\alpha)},U_2=\bigcup_\beta=(c_\beta,d_\beta)$.\par
    则$U_1\cup U_2=\left(\bigcup_\alpha{(a_\alpha,b_\alpha)}\right)\cup\left(\bigcup_\beta{(c_\beta,d_\beta)}\right) = \bigcup_{\alpha,\beta}\big((a_\alpha,b_\alpha)\cup(c_\beta,d_\beta)\big)$.\par
    其中$(a_\alpha,b_\alpha)\cup(c_\beta,d_\beta)$一定是空集或是开区间,所以$U_1\cup U_2$也是开集.\par
    对于有限多个的情形应用数学归纳法即可.\par
\end{proof}

\begin{definition}{拓扑空间}
    设$X$是一个集合,$\mathcal{T}$为$X$上的子集族,满足:
    \begin{enumerate}
        \item $\varnothing,X\in\mathcal{T}$
        \item $\mathcal{T}$中任意子集的并仍属于$\mathcal{T}$
        \item $\mathcal{T}$中有限个子集的交仍属于$\mathcal{T}$
    \end{enumerate}
    则称$\mathcal{T}$为$X$上的拓扑,$(X,\mathcal{T})$为一个拓扑空间,$\mathcal{T}$中的子集称为开集.
\end{definition}


\begin{proposition}{条件3的等价条件}
    $\mathcal{T}$中有限个子集的交仍属于$\mathcal{T} \Leftrightarrow \mathcal{T}$中两个子集的交属于$\mathcal{T}$.
\end{proposition}

\begin{instance}
    (1)在标准的实数集$\mathbb{R}$中,$\mathcal{T}:=\left\{\bigcup_\lambda{(a_\lambda,b_\lambda)}\left|(a_\lambda,b_\lambda)\text{是}\mathbb{R}\text{上的开区间}\right.\right\}$是拓扑空间.\par
    (2)$X={a,b,c}$,$\mathcal{T}_1 = \left\{\varnothing,X,\{a\},\{b\},\{a,b\}\right\}$,$\mathcal{T}_2 = \left\{\varnothing,X,\{a\},\{a,b\}\right\}$,$\mathcal{T}_3 = \left\{\varnothing,X,\{a\},\{a,c\}\right\}$.\par
    但$\left\{\varnothing,X,\{b\},\{c\},\{a,b\}\right\}$不是拓扑,因为$\{b\}\cup\{c\}=\{b,c\}$不在其中.\par
    (3)$X$是一个非空集合,$\mathcal{T}_1=\{\varnothing,X\}$,被称为\bluebf{平凡拓扑}.$\mathcal{T}_2=2^X$,被称为\bluebf{离散拓扑}.\par
    (4)$X=\mathbb{R},\mathcal{T}=\{\varnothing\cup\{\mathbb{R}\text{上有限子集的补集}\}\}$.
    \begin{enumerate}[leftmargin=4em,label=(i)]
        \item $\varnothing,\mathbb{R} \in \mathcal{T}$
        \item $\bigcup_{\lambda}{\left(\mathbb{R}\setminus F_\lambda\right)}=\mathbb{R}\setminus\left(\bigcap_{\lambda}{F_\lambda}\right) \in \mathcal{T}$
        \item $(\mathbb{R}\setminus F_1) \cap (\mathbb{R}\setminus F_2)=\mathbb{R}\setminus\left(F_1\cup F_2\right) \in \mathcal{T}$
    \end{enumerate}
    因此$\mathcal{T}$是$\mathbb{R}$上的拓扑,称为\bluebf{有限补拓扑},记作$\mathbb{R}_{fc}$.
\end{instance}

\begin{definition}{拓扑的精细与粗糙}
    设$\cal{T}_1,\cal{T}_2$是集合$X$上的两个拓扑,如果$\cal{T}_1\subset \cal{T}_2$,则称$\cal{T}_2$比$\cal{T}_1$\textbf{精细},或称$\cal{T}_1$比$\cal{T}_2$\textbf{粗糙}.
\end{definition}
\begin{conclusion}
    平凡拓扑是最粗糙的拓扑,因为它仅包含了空集和补集,是满足条件1的最低要求.\par
    离散拓扑是最精细的拓扑,因为它已经包含了集合中的所有子集.\par
\end{conclusion}



\begin{instance}
    在$\mathbb{R}$上,$\mathbb{R}_{trival}\subsetneqq\mathbb{R}_{fc}\subsetneqq\mathbb{R}_{std}\subsetneqq\mathbb{R}_{discrete}$.\par
    其中$\mathbb{R}_{fc}\subsetneqq\mathbb{R}_{std}$的原因在于任意有限集的补集都是$\mathbb{R}_{std}$中的开集,因此$\mathbb{R}_{fc}\subset \mathbb{R}_{std}$.并且如果我们取$\mathbb{R}_{std}$中的开集$(0,1)$,它的补集是一个无限集,因此不在$\mathbb{R}_{fc}$中,所以$\mathbb{R}_{fc}\subsetneqq \mathbb{R}_{std}$是严格包含的关系.
\end{instance}



\end{document}